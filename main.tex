%%%%%%%%%%%%%%%%%%%%%%%%%%%%%%%%%%%%%%%%%
% Wenneker Article
% LaTeX Template
% Version 2.0 (28/2/17)
%
% This template was downloaded from:
% http://www.LaTeXTemplates.com
%
% Authors:
% Vel (vel@LaTeXTemplates.com)
% Frits Wenneker
%
% License:
% CC BY-NC-SA 3.0 (http://creativecommons.org/licenses/by-nc-sa/3.0/)
%
%%%%%%%%%%%%%%%%%%%%%%%%%%%%%%%%%%%%%%%%%

%----------------------------------------------------------------------------------------
%	PACKAGES AND OTHER DOCUMENT CONFIGURATIONS
%----------------------------------------------------------------------------------------

\documentclass[10pt, a4paper, twocolumn]{article} % 10pt font size (11 and 12 also possible), A4 paper (letterpaper for US letter) and two column layout (remove for one column)

\input{structure.tex} % Specifies the document structure and loads requires packages

%----------------------------------------------------------------------------------------
%	ARTICLE INFORMATION
%----------------------------------------------------------------------------------------

\title{Central Operativa SAL/T} % The article title

\author{
	\authorstyle{Matías Sambrizzi \textsuperscript{1} and Fernando Iglesias\textsuperscript{1}} % Authors
	\newline\newline % Space before institutions
	\textsuperscript{1}\institution{Universidad de Buenos Aires, Buenos Aires, Argentina}\\ % Institution 1
}

% Example of a one line author/institution relationship
%\author{\newauthor{John Marston} \newinstitution{Universidad Nacional Autónoma de México, Mexico City, Mexico}}

\date{\today} % Add a date here if you would like one to appear underneath the title block, use \today for the current date, leave empty for no date

%----------------------------------------------------------------------------------------

\begin{document}

\maketitle % Print the title

\thispagestyle{firstpage} % Apply the page style for the first page (no headers and footers)

%----------------------------------------------------------------------------------------
%	ABSTRACT
%----------------------------------------------------------------------------------------

\lettrineabstract{En el presente artículo se presentan los avances del desarrollo de una arquitectura modular para la Central Operativa SAL/T basado en el paradigma de Internet de las Cosas (IoT). Este proyecto es el complemento en materia de \textit{software} de un prototipo realizado para Trenes Argentinos, entidad que se ocupa de gestionar y operar la red ferroviaria de la Argentina.}

\lettrineabstract{Keywords: MQTT, JSON, ...}


%----------------------------------------------------------------------------------------
%	ARTICLE CONTENTS
%----------------------------------------------------------------------------------------

\section{Motivación y Contexto}

Las formaciones ferroviarias cuentan con diferentes sistemas de seguridad a bordo, siendo estos, equipos que se encargan de supervisar el correcto funcionamiento de los subsistemas críticos. Ante una falla en uno de los subsistemas, una formación ferroviaria se detiene inmediatamente por la activación automática de las señales de corte de tracción (\textit{CT}) y frenado de emergencia (\textit{FE}). En esta situación, el conductor debe llevar la formación a un lugar seguro para que los pasajeros puedan descender y, posteriormente, trasladarla a un taller para que pueda ser reparada.

El \textit{SAL/T} \footnote{referencia al artículo de Di Vito}, según sus siglas, Sistema de Aislamiento Limitado o Total, es un sistema que le permite al maquinista de una formación ferroviaria la posibilidad de activar y desactivar del modo aislado limitado. En este modo, el equipo permite la circulación de la formación al desactivar las señales de corte de tracción y freno de emergencia generadas por los otros subsistemas. Para que esta operación se realice de forma segura, se debe monitorear la velocidad de la formación tal que sea posible evitar que supere cierto valor máximo. 

A partir del prototipo físico del SAL/T previamente mencionado, se presentan los avances en el desarrollo de una central operativa, es una plataforma web que cuenta con una unidad lógica de compartición y empaquetado de software, que posibilita la administación, la configuración y el monitoreo en tiempo real de la información recibida y transmitida por parte de cada dispositivo SAL/T.

\section{Descripción de la problemática a resolver}

Los subsistemas asociados al SAL/T como la seguridad de puertas, el sistema de hombre vivo y la protección de coche a la deriva; son críticos debido a que, en caso de fallar, pueden ocasionar lesiones o muertes de personas, dañar el medio ambiento e inclusdo generar pérdidas materiales. 

En efecto, la central operativa permite la administación y configuración de forma remota los dispositivos de supervisión de seguridad de cada formación ferroviaria, la visualización de los diferentes parámetros de interés involucrados tal que se encuentre al alcance de una o más personas asignadas dentro de una entidad y de este modo, sea posible optimizar la toma de decisiones.

Como consecuencia de lo expresado previamente, se ha realizado el diseño de una arquitectura versátil y confiable en términos de capacidad de procesamiento y de almacenamiento de la información.

En la sección III, según el \textit{stack} de una solución IoT, se presentan cada una de las tecnologías utilizadas. Luego, en la sección IV se exponen las conclusiones y por último, en la sección V, se presentan los próximos pasos del trabajo.


\section{Arquitectura propuesta}

En el marco del presente trabajo, la arquitectura esta estratificada en capas
tal como se ilustra en la fiugra X. A continuación se describen cada una de ellas

\subsection{Capa de dispositivos}

\subsection{Capa de conectividad}

\subsection{Capa de almacenamiento}

La capa de almacenamiento por un lado se encuentra constituida por el sistema distribuido \textit{Kafka} que se encarga
de almacenar los datos de la aplicación. Entre ellos se destacan los mensajes MQTT enviados
por los dispositivos SAL/T, como también la información referida a las formaciones y a los SAL/T que las ocupan. 
Ademas, se tienen servicios de procesamiento de datos que se encargan de adaptar la información de los tópicos en
datos que posteriormente serán consumidos por los servicios que integran la capa superior. Por otro lado,
se tiene una base de datos SQL que se utiliza para almacenar los datos del servicio de autenticación y 
Hasura. Los conectores se encargan de insertar los datos que se quieran disponer a la capa de 
visualización.

\subsection{Capa de interacción}

La capa de interacción se encarga de proveer la autenticación, la consultas e inserción de datos en el sistema.
La misma, esta constituida por un backend y un servicio de autenticación desarrollados en Kotlin utilizando el framework Ktor y
Hasura que se encarga de generar la API de consultar a la base de datos. Este herramienta, 
facilita la comunicación entre el frontend y la base de datos SQL. 

\subsection{Capa de visualizacion}

En la capa de visualización se desarrollo una aplicación web utilizando el lenguaje
de programación Typescript y el framework React.

\section{Conclusiones}

En este trabajo se presentan las principales características del diseño de una arquitectura en materia de \textit{software} y de \textit{firmware} para la operabilidad de los dispositivos SAL/T.

Entre los módulos desarrollados, se posibilita la visualización de los usuarios y los dispositivos de cada entidad ofreciendo una solución sencilla y escalable. También, se cuenta la ingestión de la información recibida desde los dispositivos permitiendo una alta disponibilidad y operabilidad de los recursos. Por último, se cuenta con un microservicio destinado a la gestión de usuarios y perfiles que posee una alta tolerancia a fallos y adaptable.


\section{Próximos pasos}

En la actualidad, se encuentra en desarrollo propiciar al operario enviar comandos de control a los dispositivos activos que tenga asignados, brindar la connfiguración de los parámetros de cada dispositivo y de ser deseable su correspondiente modificación y finalmente, brindar la seguridad en las comunicaciones agregando la capa \textit{TLS} en el protocolo \textit{MQTT}. 



This sentence requires citation \citep{Reference1}. This sentence requires multiple citations to imply that it is better supported \citep{Reference2,Reference3}. Finally, when conducting an appeal to authority, it can be useful to cite a reference in-text, much like \cite{Reference1} do quite a bit. Oh, and make sure to check out the bear in Figure \ref{bear}.

%----------------------------------------------------------------------------------------
%	BIBLIOGRAPHY
%----------------------------------------------------------------------------------------

\printbibliography[title={Bibliography}] % Print the bibliography, section title in curly brackets

%----------------------------------------------------------------------------------------

\end{document}
