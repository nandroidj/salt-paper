\documentclass[a4paper]{IEEEtran}
\IEEEoverridecommandlockouts
% The preceding line is only needed to identify funding in the first footnote. If that is unneeded, please comment it out.
\usepackage{}
\usepackage{amsmath,amssymb,amsfonts}
\usepackage{algorithmic}
\usepackage{graphicx}
\usepackage{textcomp}
\usepackage{xcolor}
\usepackage[font=normalsize,labelfont=bf]{caption}
\usepackage{balance}



\def\BibTeX{{\rm B\kern-.05em{\sc i\kern-.025em b}\kern-.08em
    T\kern-.1667em\lower.7ex\hbox{E}\kern-.125emX}}


\begin{document}

% Opcion 1
\title{Diseño e implementación de una central operativa para el control y monitoreo en el material rodante}


\author{

xxxxxx xxxxxxxxx, 
xxxxxxxx xxxxxxxx, 
xxxxx xxxxxxxxx \\

\normalsize \textit{xxxxxxx xxxxxxxx} \\
xxxxxxxxxxx xx xxxxxxxx xxxxxxxxx \\
\textit{xxxxxxxx xx xxxxxxxxxx - xxx} \\
\textit{xxxxxx xxxxx - xxxxxxxxx}
\textsuperscript{1}\small xxxxxxxxxxxxx \\

}


\maketitle


\begin{abstract}
\\
Ante la falla de sistemas como el control de puertas o la cámara frontal de registro de accidentes viales las formaciones ferroviarias activan automáticamente el corte de tracción y el freno de emergencia. En esas circunstancias la formación queda detenida y los pasajeros deben descender a las vías si la detención no se produce en una estación ferroviaria. En este trabajo se presenta el desarrollo para Trenes Argentinos de una arquitectura modular basada en el paradigma de Internet de las Cosas (IoT) que permite visualizar y gestionar los sistemas involucrados en esas situaciones.
\end{abstract}


\begin{IEEEkeywords}
\\
Sistemas ferroviarios, MQTT/TLS, TypeScript, Apache Kafka
\end{IEEEkeywords}



\section{Motivación y Contexto}

El sistema ferroviario de la República Argentina cuenta con una gran cantidad de formaciones ferroviarias en las que se encuentran diferentes sistemas de seguridad a bordo. Estos equipos se encargan de supervisar el correcto funcionamiento de los subsistemas críticos. Ante una falla en uno de los subsistemas, una formación ferroviaria se detiene inmediatamente por la activación automática de las señales de corte de tracción (\textit{CT}) y frenado de emergencia (\textit{FE}). En esta situación, el conductor debe llevar la formación a un lugar seguro para que los pasajeros puedan descender y, posteriormente, trasladarla a un taller para que pueda ser reparada.

El \textit{SAL/T}, según sus siglas, Sistema de Aislamiento Limitado / Total \cite{b1}, es un dispositivo del cual se cuenta con una primera versión prototipada; que se presenta como solución a las contingencias descritas anteriormente. De esta manera, el maquinista de una formación ferroviaria cuenta con la posibilidad de activar y desactivar el \textit{modo aislado limitado}. En este modo, el equipo permite la circulación de la formación al desactivar las señales de corte de tracción y freno de emergencia generadas por los subsistemas críticos. Para que esta operación se complete de forma segura, se debe monitorear la velocidad de la formación tal que sea posible garantizar que no se supere cierto valor máximo.

A partir del desarrollo previo del prototipo, se presentan los avances en la implementación de una central operativa que centraliza los dispositivos SAL/T para su respectiva administración, configuración y monitoreo en tiempo real de la información recibida y transmitida desde una plataforma digital. El proyecto se desarrolla por xxxxxxx-xxxxxxx \cite{b2} para la empresa Trenes Argentinos \cite{b3}.

Los subsistemas asociados al \textit{SAL/T}, como la seguridad de puertas, el sistema de hombre vivo y la protección de coche a la deriva, son críticos debido a que, en caso de fallar, pueden ocasionar lesiones o muertes de personas e incluso generar pérdidas materiales. 

La central operativa permite la administración y configuración en forma remota de los dispositivos de supervisión de seguridad de cada formación ferroviaria, la visualización de los diferentes parámetros de interés involucrados por las personas asignadas dentro de una entidad y de este modo es posible optimizar la toma de decisiones. 

La sección II presenta cada una de las capas con sus respectivas tecnologías en basándose en el \textit{stack IoT} \cite{b4}. Luego, en la sección III se exponen las conclusiones del trabajo.




\section{Descripción de la problemática a resolver}

Los subsistemas asociados al \textit{SAL/T}, como la seguridad de puertas, el sistema de hombre vivo y la protección de coche a la deriva, son críticos debido a que, en caso de fallar, pueden ocasionar lesiones o muertes de personas e incluso generar pérdidas materiales. 

La central operativa permite la administración y configuración en forma remota de los dispositivos de supervisión de seguridad de cada formación ferroviaria, la visualización de los diferentes parámetros de interés involucrados por las personas asignadas dentro de una entidad y de este modo, sea posible optimizar la toma de decisiones. 

A partir de los modelos más relevantes que se encuentran establecidos para el desarrollo e implementación de aplicaciones de software se consideran el \textit{stack web} y el \textit{stack IoT} \cite{b4}. 
Dadas las similitudes entre las capas de ambos esquemas, excepto en la capa de aplicación, donde el stack web emplea el protocolo \textit{HTTP} \cite{b5} mientras que el \textit{stack IoT} utiliza el protocolo \textit{MQTT} \cite{b6}; siendo este último el más adecuado para escenarios donde son limitados los recursos como el ancho de banda y el consumo de energía.

En especial, el protocolo \textit{MQTT} dispone de mensajes más livianos y además es posible transmitir y/o recibir datos en formato binario sin la necesidad de una codificación previa. También, este protocolo permite asignar niveles de calidad de servicio (\textit{QoS}) \cite{b7} a los mensajes transmitidos, resultando una característica primordial en aplicaciones donde la probabilidad de pérdidas de paquetes es considerable.

La sección III presenta cada una de las capas con sus respectivas tecnologías en base al stack descripto anteriormente. Luego, en la sección IV se exponen las conclusiones del trabajo.





\section{Arquitectura propuesta}

En el marco del presente trabajo, la arquitectura esta estratificada en capas
tal como se ilustra en la figura 1. En las siguientes subsecciones se describen cada una de estas capas.

\begin{figure}[ht]
\centering 
\includegraphics[width=.48\textwidth]{images/iot-stack-v2.1.png}
\caption{Arquitectura de la Central Operativa SAL/T.}
\label{fig:diagBloques}
\end{figure}

\subsection{Capa de dispositivos}

Sobre la base de un prototipo del Sistema de Aislamiento Limitado o Total, \textit{SAL/T}, se desarolló, en el kit de desarrollo \textit{Nucleo F429} \cite{b5}, un cliente \textit{MQTT} \cite{b6} que permite la publicación de la velocidad de la formación y de los parámetros provistos por los subsistemas de falla, entre otras variables; y llegado el caso, el dispositivo pueda realizar la lectura de diferentes parámetros configurables.

\subsection{Capa de conectividad}

Para la comunicación entre los dispositivos \textit{SAL/T} y la capa de almacenamiento se utiliza un \textit{Broker MQTT} que oficia de orquestador entre el cliente, el dispositivo \textit{SAL/T} que publica los mensajes y \textit{Apache Kafka} \cite{b7}, un sistema distribuido que realiza la lectura de la información y su posterior procesamiento. 

Dado que los mensajes intercambiados entre las partes contienen información sensible, se ha optado por agregar la capa de seguridad \textit{TLS} \cite{b8}. En este sentido, resulta indispensable utilizar el certificado \textit{X509} \cite{b9} para prevenir ataques del tipo \textit{adversary-in-the-middle} \cite{b10} y utilizar certificados emitidos por autoridades reconocidas.


\subsection{Capa de almacenamiento}

La capa de almacenamiento se encuentra constituida por el sistema distribuido \textit{Kafka} que se encarga de almacenar los datos de la aplicación. Entre ellos se destacan los mensajes MQTT enviados por los dispositivos SAL/T, como también la información referida a las formaciones y a los SAL/T que las ocupan. Además, se tienen servicios de procesamiento de datos que se encargan de adaptar la información de los tópicos en datos que posteriormente serán consumidos por los servicios que integran la capa superior.

Por otro lado, se tiene una base de datos relacional \textit{PostgreSQL} \cite{b11} que se utiliza para almacenar los datos del servicio de autenticación y el motor que facilita la interacción con la plataforma web. Los conectores se encargan de insertar los datos que se quieran disponer a la capa de visualización.

\begin{figure}[ht]
\centering 
\includegraphics[width=.5\textwidth]{images/salt_table_v2.1.jpg}
\caption{Modelo de datos de los dispositivos, trenes y entidades.}
\label{fig:devicesTrainEntity}
\end{figure}

\begin{figure}[ht]
\centering 
\includegraphics[width=.5\textwidth]{images/fail_system_v2.1.jpg}
\caption{Modelo de datos de los dispositivos y sus subsistemas de fallas.}
\label{fig:devicesSubsystems}
\end{figure}

En la figura \ref{fig:devicesTrainEntity} se observa el modelo de datos que representa a los dispositivos SAL/T, las formaciones y las entidades. 
El primero cuenta con un identificador del dispositivo en el sistema, el estado de operación y una referencia al tren en el que se desplegó el artefacto. 
El modelo de la formación está enriquecido con la entidad a la que pertenece y el numero de serie del tren. 
Por último, las entidades están compuestas por nombre y descripción.
Luego, en la imagen \ref{fig:devicesSubsystems} se aprecian los subsistemas de fallas relacionados con el dispositivo SAL/T.
Su modelo esta integrado con el nombre del subsistema, el id del dispositivo al que está asociado y el estado actual del mismo. 

\subsection{Capa de interacción}

La capa de interacción se encuentra conformada por tres unidades. Por un lado, se ha desarrollado en el lenguaje \textit{Kotlin} \cite{b12} en conjunto con el \textit{framework} \textit{Ktor} \cite{b13}, un microservicio de autenticación donde es posible la gestión de los usuarios con sus respectivos roles y también el \textit{backend}. Entre sus tareas principales se encuentran las relacionadas al protocolo \textit{MQTT} para la actualización de los certificados que brindan seguridad a las comunicaciones, mediante el uso de la capa de seguridad \textit{TLS}, y la indexación de los eventos que se publiquen en tiempo real; como la configuración general que permite el funcionamiento integral de los servicios y módulos dispuestos en el sistema. \\

Además, se dispone del motor \textit{Hasura} \cite{b14}, el cual se encuentra conectado a una base de datos relacional (\textit{PostgreSQL}) basado en el lenguaje de consulta estructurado \textit{SQL} \cite{b15}, que permite exponer una \textit{API GraphQL} \cite{b16} a aquellos clientes que deseen obtener una visualización del modelo de datos propuesto en que se conjuga la información de cada formación ferroviaria con su correspondiente dispositivo \textit{SAL/T} y diseño destinado al \textit{frontend}. Más aún, el enlace permite realizar modificaciones y hasta remociones de las columnas propuestas en las tablas de cada esquema dentro de la base de datos.


\subsection{Capa de visualización}

En esta capa, se ha utilizado el lenguaje de programación \textit{TypeScript} \cite{b17} junto con la biblioteca gráfica \textit{React} \cite{b18} para brindar una web reactiva en la que se elaboraron los formularios, las tablas y los paneles a partir de la explotación en de los datos almacenados en la base de datos. Cabe destacar que la plataforma web se encuentra condicionada por el rol y la entidad ferroviaria a la que pertenece cada usuario.

En la figura \ref{fig:dashboard} se puede apreciar el dashboard de la central operativa al que tendrán acceso los operadores de cada entidad.
En el mismo se observan distintas tarjetas con el estado de cada subsistema de fallas y del tren, como así también un velocímetro que
informa la velocidad de la formación.

\begin{figure}[ht]
\centering 
\includegraphics[width=.47\textwidth]{images/cental.png}
\caption{Panel de control de la central operativa.}
\label{fig:dashboard}
\end{figure}



\section{Conclusiones}

Se presentaron las principales características del diseño de una arquitectura en materia de \textit{software} y de \textit{firmware} para la operabilidad de los dispositivos SAL/T.

Mediante los módulos desarrollados se posibilita la visualización de los usuarios y los dispositivos de cada entidad ofreciendo una solución sencilla y escalable. También, se cuenta la ingestión de la información recibida desde los dispositivos permitiendo una alta disponibilidad y operabilidad de los recursos. Por último, se cuenta con un microservicio destinado a la gestión de usuarios y perfiles que posee una alta tolerancia a fallos y adaptable.

En la actualidad, se encuentra en desarrollo propiciar al operario enviar comandos de control a los dispositivos activos que tenga asignados, brindar la configuración de los parámetros de cada dispositivo y de ser deseable su correspondiente modificación y finalmente, brindar la seguridad en las comunicaciones agregando la capa \textit{TLS} en el protocolo \textit{MQTT}.


% 

\begin{thebibliography}{00}

\bibitem{b1} 'Sistema de supervisión de la seguridad del material
ferroviario utilizando patrones de diseño', Ivan Mariano Di Vito, Pablo Gomez, Ariel Lutenberg, Libro de trabajos del CASE2019, Congreso Argetino de Sistemas Embebidos, Santa Fe, Argentina. (2019).

\bibitem{b2} CONICET-GICSAFe (2022). [Online]. Disponible: \\
https://sites.google.com/view/conicet-gicsafe/inicio

\bibitem{b3} Trenes Argentinos (2022). [Online]. Disponible: \\
https://www.argentina.gob.ar/transporte/trenes 

\bibitem{b4} Introducción general a IoT (2022). [Online]. Disponible: \\
https://www.gotoiot.com/pages/articles/iot\_intro/index.html 

\bibitem{b5} NUCLEO-F429ZI (2022). [Online]. Disponible: \\
https://www.st.com/en/evaluation-tools/nucleo-f429zi.html

\bibitem{b6} MQTT (2022). [Online]. Disponible: https://mqtt.org/ 

\bibitem{b7} Kafka (2022). [Online]. Disponible: https://kafka.apache.org/  

\bibitem{b8} 'TLS/SSL - MQTT Security Fundamentals' (2022). [Online]. \\
Disponible: https://www.hivemq.com/blog/mqtt-security-fundamentals-tls-ssl/ 

\bibitem{b9} X509 (2022). [Online]. Disponible: \\
https://www.ssl.com/faqs/what-is-an-x-509-certificate/ 

\bibitem{b10} Adversary-in-the-Middle (2022). [Online]. Disponible: \\
https://attack.mitre.org/techniques/T1557/

\bibitem{b11} PostgreSQL (2022). [Online]. Disponible: https://www.postgresql.org/

\bibitem{b12} Kotlin (2022). [Online]. Disponible: https://kotlinlang.org/

\bibitem{b13} Ktor (2022). [Online]. Disponible: https://ktor.io/ 

\bibitem{b14} Hasura (2022). [Online]. Disponible: https://hasura.io/

\bibitem{b15} SQL (2022). [Online]. Disponible: https://www.w3schools.com/sql/

\bibitem{b16} API GraphQL (2022). [Online]. Disponible: https://graphql.org/

\bibitem{b17} TypeScript (2022). [Online]. Disponible: https://www.typescriptlang.org/

\bibitem{b18} React (2022). [Online]. Disponible: https://reactjs.org/


\end{thebibliography}



\balance
\begin{thebibliography}{00}

\bibitem{b1} 'Sistema de supervisión de la seguridad del material
ferroviario utilizando patrones de diseño', Ivan Mariano Di Vito, Pablo Gomez, Ariel Lutenberg, Libro de trabajos del CASE2019, Congreso Argetino de Sistemas Embebidos, Santa Fe, Argentina. (2019).

\bibitem{b2} CONICET-GICSAFe (June 2022). [Online]. Disponible: \\ https://sites.google.com/view/conicet-gicsafe/inicio

\bibitem{b3} Trenes Argentinos (June 2022). [Online]. Disponible: \\ https://www.argentina.gob.ar/transporte/trenes 

\bibitem{b4} 'Internet of Things for Measuring Human Activities in Ambient Assisted Living and e-Health', Amine Rghioui, Sendra Sandra, Lloret Jaime, Oumnad Abedlmajid,
Network Protocols and Algorithms. (2016).

\bibitem{b5} HTTP (June 2022). [Online]. Disponible: \\ https://developer.mozilla.org/es/docs/Web/HTTP 

\bibitem{b6} MQTT (June 2022). [Online]. Disponible: https://mqtt.org/ 

\bibitem{b7} QoS (June 2022). [Online]. Disponible: \\ https://www.hivemq.com/blog/mqtt-essentials-part-6-mqtt-quality-of-service-levels/

\bibitem{b8} NUCLEO-F429ZI (June 2022). [Online]. Disponible: \\ https://www.st.com/en/evaluation-tools/nucleo-f429zi.html

\bibitem{b9} Kafka (June 2022). [Online]. Disponible: https://kafka.apache.org/  

\bibitem{b10} 'TLS/SSL - MQTT Security Fundamentals' (2022). [Online]. \\ Disponible: https://www.hivemq.com/blog/mqtt-security-fundamentals-tls-ssl/ 

\bibitem{b11} X509 (June 2022). [Online]. Disponible: \\ https://www.ssl.com/faqs/what-is-an-x-509-certificate/ 

\bibitem{b12} Adversary-in-the-Middle (June 2022). [Online]. Disponible: \\ https://attack.mitre.org/techniques/T1557/

\bibitem{b13} PostgreSQL (June 2022). [Online]. Disponible: \\ https://www.postgresql.org/

\bibitem{b14} Kotlin (June 2022). [Online]. Disponible: https://kotlinlang.org/

\bibitem{b15} Ktor (June 2022). [Online]. Disponible: https://ktor.io/ 

\bibitem{b16} Hasura (June 2022). [Online]. Disponible: https://hasura.io/

\bibitem{b17} SQL (June 2022). [Online]. Disponible: https://www.w3schools.com/sql/

\bibitem{b18} API GraphQL (2022). [Online]. Disponible: https://graphql.org/

\bibitem{b19} TypeScript (June 2022). [Online]. Disponible: \\ https://www.typescriptlang.org/

\bibitem{b20} React (June 2022). [Online]. Disponible: https://reactjs.org/


\end{thebibliography}

\end{document}




