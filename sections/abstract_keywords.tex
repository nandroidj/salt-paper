
\begin{abstract}
\\
Ante la falla de sistemas tales como el control de puertas o la cámara frontal de registro de accidentes viales las formaciones ferroviarias activan automáticamente el corte de tracción y el freno de emergencia. En esas circunstancias la formación queda detenida y los pasajeros deben descender a las vías si la detención no se produce en una estación ferroviaria. En este trabajo se presenta el desarrollo para Trenes Argentinos de una arquitectura modular basada en el paradigma de Internet de las Cosas (IoT) que permite visualizar y gestionar los sistemas involucrados en esas situaciones.
\end{abstract}


\begin{IEEEkeywords}
\\
Sistemas ferroviarios, MQTT/TLS, TypeScript, Apache Kafka
\end{IEEEkeywords}
