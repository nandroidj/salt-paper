
\section{Conclusiones}

Se presentaron las principales características del diseño de una arquitectura en materia de \textit{software} y de \textit{firmware} para la operabilidad de los dispositivos SAL/T.

Mediante los módulos desarrollados se posibilita la visualización de los usuarios y los dispositivos de cada entidad ofreciendo una solución sencilla y escalable. También, se cuenta la ingestión de la información recibida desde los dispositivos permitiendo una alta disponibilidad y operabilidad de los recursos. Por último, se cuenta con un microservicio destinado a la gestión de usuarios y perfiles que posee una alta tolerancia a fallos y adaptable.

En la actualidad, se encuentra en desarrollo propiciar al operario enviar comandos de control a los dispositivos activos que tenga asignados, brindar la configuración de los parámetros de cada dispositivo y de ser deseable su correspondiente modificación y finalmente, brindar la seguridad en las comunicaciones agregando la capa \textit{TLS} en el protocolo \textit{MQTT}.
