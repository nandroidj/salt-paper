
\section{Conclusiones}

Los avances obtenidos permiten confirmar que el diseño propuesto ha sido el apropiado. 

La arquitectura implementada facilitó la disponibilidad y la explotación de la información proveniente de los dispositivos SAL/T, así como del microservicio correspondiente a la autenticación de usuarios.
De este modo, se posibilitó la visualización de los integrantes y los dispositivos de cada entidad ofreciendo una solución sencilla y escalable. 

En la actualidad, se encuentra en desarrollo propiciar al operario el envio de comandos para el control de los dispositivos activos que este tenga asignados, actualizar la configuración de los parámetros de cada dispositivo
y finalmente, brindar la seguridad en las comunicaciones agregando la capa \textit{TLS} en el protocolo \textit{MQTT}.
