
\section{Motivación y Contexto}

El sistema ferroviario de la República Argentina cuenta con una gran cantidad de formaciones ferroviarias en las que se encuentran diferentes sistemas de seguridad a bordo. Estos equipos se encargan de supervisar el correcto funcionamiento de los subsistemas críticos. Ante una falla en uno de los subsistemas, una formación ferroviaria se detiene inmediatamente por la activación automática de las señales de corte de tracción (\textit{CT}) y frenado de emergencia (\textit{FE}). En esta situación el conductor debe llevar la formación a un lugar seguro para que los pasajeros puedan descender y, posteriormente, trasladarla a un taller para que pueda ser reparada.

El \textit{SAL/T} \cite{b1}, según sus siglas, Sistema de Aislamiento Limitado o Total, es un dispositivo del cual se cuenta con una primera versión prototipada; que se presenta como solución a las contingencias descriptas anteriormente. De esta manera, el maquinista de una formación ferroviaria cuenta con la posibilidad de activar y desactivar el modo ailsado limitado.     

En el modo asilado limitado, el equipo permite la circulación de la formación al desactivar las señales de corte de tracción y freno de emergencia generadas por los subsistemas críticos. Para que esta operación se complete de forma segura, se debe monitorear la velocidad de la formación tal que sea posible evitar que supere cierto valor máximo.

A partir del desarrollo previo del prototipo, se presentan los avances en la implementación de una central operativa que centraliza los dispositivos SAL/T para su respectiva administación, configuración y monitoreo en tiempo real de la información recibida y transmitida desde una plataforma digital. El proyecto se desarrolla por xxxxxxx-xxxxxxx \cite{b2} para la empresa Trenes Argentinos \cite{b3}.
