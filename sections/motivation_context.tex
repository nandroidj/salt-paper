
\section{Motivación y Contexto}

El sistema ferroviario de la República Argentina cuenta con una gran cantidad de formaciones ferroviarias en las que se encuentran diferentes sistemas de seguridad a bordo. Estos equipos se encargan de supervisar el correcto funcionamiento de los subsistemas críticos. Ante una falla en uno de los subsistemas, una formación ferroviaria se detiene inmediatamente por la activación automática de las señales de corte de tracción (\textit{CT}) y frenado de emergencia (\textit{FE}). En esta situación, el conductor debe llevar la formación a un lugar seguro para que los pasajeros puedan descender y, posteriormente, trasladarla a un taller para que pueda ser reparada.

El \textit{SAL/T}, según sus siglas, Sistema de Aislamiento Limitado / Total \cite{b1}, es un dispositivo del cual se cuenta con una primera versión prototipada; que se presenta como solución a las contingencias descritas anteriormente. De esta manera, el maquinista de una formación ferroviaria cuenta con la posibilidad de activar y desactivar el \textit{modo aislado limitado}. En este modo, el equipo permite la circulación de la formación al desactivar las señales de corte de tracción y freno de emergencia generadas por los subsistemas críticos. Para que esta operación se complete de forma segura, se debe monitorear la velocidad de la formación tal que sea posible garantizar que no se supere cierto valor máximo.

A partir del desarrollo previo del prototipo, se presentan los avances en la implementación de una central operativa que centraliza los dispositivos SAL/T para su respectiva administación, configuración y monitoreo en tiempo real de la información recibida y transmitida desde una plataforma digital. El proyecto se desarrolla por xxxxxxx-xxxxxxx \cite{b2} para la empresa Trenes Argentinos \cite{b3}.

Los subsistemas asociados al \textit{SAL/T}, como la seguridad de puertas, el sistema de hombre vivo y la protección de coche a la deriva, son críticos debido a que, en caso de fallar, pueden ocasionar lesiones o muertes de personas e incluso generar pérdidas materiales. 

La central operativa permite la administración y configuración en forma remota de los dispositivos de supervisión de seguridad de cada formación ferroviaria, la visualización de los diferentes parámetros de interés involucrados por las personas asignadas dentro de una entidad y de este modo es posible optimizar la toma de decisiones. 

La sección II presenta cada una de las capas con sus respectivas tecnologías en basándose en el \textit{stack IoT} \cite{b4}. Luego, en la sección III se exponen las conclusiones del trabajo.


