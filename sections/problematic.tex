
\section{Descripción de la problemática a resolver}

Los subsistemas asociados al SAL/T, como la seguridad de puertas, el sistema de hombre vivo y la protección de coche a la deriva, son críticos debido a que, en caso de fallar, pueden ocasionar lesiones o muertes de personas e incluso generar pérdidas materiales. \\ 

La central operativa permite la administración y configuración en forma remota de los dispositivos de supervisión de seguridad de cada formación ferroviaria, la visualización de los diferentes parámetros de interés involucrados por las personas asignadas dentro de una entidad y de este modo, sea posible optimizar la toma de decisiones. \\

Como consecuencia de lo expresado previamente, se ha realizado el diseño de una arquitectura de \textit{software} versátil y confiable en términos de capacidad de procesamiento y de almacenamiento de la información. \\

En la sección III, basándose en el \textit{stack} de una solución \textit{IoT} \cite{b4} se presentan cada una de las capas con sus respectivas tecnologías. Luego, en la sección IV se exponen las conclusiones y los próximos pasos del trabajo.

